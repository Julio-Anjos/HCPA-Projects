\documentclass[cic,tc]{iiufrgs}
\usepackage[utf8]{inputenc}   % pacote para acentuação
\usepackage{graphicx}         % pacote para importar figuras
\usepackage{times}            % pacote para usar fonte Adobe Times
\usepackage{algpseudocode}
\usepackage{url}
\usepackage{amsmath}
% \usepackage{palatino}
% \usepackage{mathptmx}       % p/ usar fonte Adobe Times nas fórmulas
\usepackage[alf,abnt-emphasize=bf]{abntex2cite}	% pacote para usar citações abnt

%
% Informações gerais
%
\title{Estudo e otimização dos softwares de bioinformática do Hospital de
Clínicas de Porto Alegre}

\author{Farah}{Alef}

% orientador e co-orientador são opcionais (não diga isso pra eles :))
\advisor[Prof.~Dr.]{Geyer}{Claudio Fernando Resin}
\coadvisor[Prof.~Dr.]{Anjos}{Julio Cesar Santos}

% a data deve ser a da defesa; se nao especificada, são gerados
% mes e ano correntes
\date{novembro}{2021}

% o local de realização do trabalho pode ser especificado (ex. para TCs)
% com o comando \location:
\location{Porto Alegre}{RS}

% itens individuais da nominata podem ser redefinidos com os comandos
%
% palavras-chave
% iniciar todas com letras minúsculas, exceto no caso de abreviaturas
%
\keyword{bioinformática}
\keyword{paralelismo}
\keyword{codeml}
\keyword{PAML}
\keyword{SAMtools}
%\keyword{}

%\settowidth{\seclen}{1.10~}

\begin{document}

% folha de rosto
% às vezes é necessário redefinir algum comando logo antes de produzir
% a folha de rosto:
% \renewcommand{\coordname}{Coordenadora do Curso}
\maketitle

% dedicatoria
% \clearpage
% \begin{flushright}
%     \mbox{}\vfill
%     {\sffamily\itshape
%       ``If I have seen farther than others,\\
%       it is because I stood on the shoulders of giants.''\\}
%     --- \textsc{Sir~Isaac Newton}
% \end{flushright}

% agradecimentos
%\chapter*{Agradecimentos}
%Agradeço ao \LaTeX\ por não ter vírus de macro\ldots

% resumo na língua do documento
\begin{abstract}
  Neste trabalho foi realizado um estudo e otimização dos problemas de
  desempenho presentes nos softwares de bioinformática utilizados pelo grupo de
  pesquisa em genética do Hospital de Clínicas de Porto Alegre (HCPA),
  empregando para isso técnicas de processamento paralelo. O trabalho focou
  no software de análise filogenética codeml, do pacote PAML, amplamente
  utilizado na literatura, e no software SAMtools, usado para chamda de
  variantes, igualmente popular. Foi desenvolvida uma ferramenta de
  paralelização de \textit{jobs} do SAMtools, reduzido de vários dias para
  poucas horas o tempo de análise do grupo de pesquisa, enquanto que no caso do
  codeml foi realizado uma análise de desempenho de alternativas encontradas em
  revisão bibliográfica, fornecendo aos pesquisadores uma ferramenta que reduz
  em mais da metade o tempo de execução em relação ao codeml original.
\end{abstract}

% resumo na outra língua
% como parametros devem ser passados o titulo e as palavras-chave
% na outra língua, separadas por vírgulas
\begin{englishabstract}{Study and optimization of the bioinformatics software used by Hospital de Clínicas de Porto Alegre}{Bioinformatics, parallelism, codeml, PAML, SAMtools} In this paper we studied and optimized the performance bottlenecks found in the bioinformatics software used by the genetics research group from Hospital de Clínicas de Porto Alegre (HCPA), employing parallel programming techniques to achieve these goals. The focus of our work was in the phylogenetic analysis software called codeml, from the PAML package, which is widely used in the literature, as well on the SAMtools software, used for variant calling, also popular. We developed a tool for the parallel execution of SAMtools jobs, reducing total execution time for the group's input from various days to a few hours, while in the case of codeml we analyzed the performance of solutions found in a bibliography review, supplying the researches with a tool whose execution time is half that of codeml.
\end{englishabstract}

% lista de figuras
\listoffigures

% lista de tabelas
\listoftables

% lista de abreviaturas e siglas
% o parametro deve ser a abreviatura mais longa
\begin{listofabbrv}{HCPA}
    \item[HCPA] Hospital de Clínicas de Porto Alegre
    \item[PAML] \textit{Phylogenetic Analysis By Maximum Likelihood}
    \item[FTCS] \textit{Forward Time Centered Space}
    \item[CPU] \textit{Central Processing Unit}
    \item[GPU] \textit{Graphics Processing Unit}
    \item[BFGS] Broyden–Fletcher–Goldfarb–Shanno
    \item[DNA] Ácido desoxirribonucleico
    \item[RNA] Ácido ribonucleico
    \item[RAM] \textit{Random access memory}
    \item[GB] Giga byte
    \item[GHz] Giga hertz
    \item[ARC] \textit{Advanced Resource Connector}
    \item[PBS] \textit{Population Branch Statistic}
\end{listofabbrv}

% idem para a lista de símbolos
\begin{listofsymbols}{$\omega$}
    \item[$\omega$] Taxa de substituição
    \item[$dN$] Substituições não-sinônimas
    \item[$dS$] Substituições sinônimas
    \item[$F_{ST}$] Índice de fixação
    \item[$\frac{\partial f}{\partial x_i}$] Derivada parcial de $f$ com respeito a $x_i$
\end{listofsymbols}

% sumario
\tableofcontents

% aqui comeca o texto propriamente dito

% introducao
\chapter{Introdução}

Neste trabalho foi realizado um estudo e otimização dos problemas de desempenho
presentes nos softwares de bioinformática utilizados pelo grupo de pesquisa em
genética do Hospital de Clínicas de Porto Alegre (HCPA), empregando para isso
técnicas de processamento paralelo e distribuído sempre que viável.

Nomeadamente foram estudados os softwares codeml, do pacote
PAML,\cite{yang2007paml} amplamente utilizado na literatura para análise
filogenética;\cite{maldonado2016lmap} o pacote ANGSD, utilizado para comparação
de sequências genéticas de diferentes populações de uma
espécie;\cite{korneliussen2014angsd} e o pacote SAMtools, amplamente utilizado
na literatura para manipulação de arquivos de sequências genéticas
alinhadas e chamada de variantes,\cite{danecek2021twelve} e usado pelo grupo
como uma alternativa ao software ANGSD.

Cada um desses softwares apresentava problemas de desempenho diferentes dentro
do pipeline de análise genética do grupo de pesquisa. Nomeadamente, o PAML e o
SAMtools apresentavam tempos de execução proibitivamente elevados, enquanto que
o ANGSD apresentava uso de memória proibitivamente alto. No caso dos softwares
PAML e ANGSD foram realizadas análises de desempenho dos softwares originais
bem como de alternativas encontradas através de um estudo da literatura,
enquanto que no caso do pacote SAMtools foi desenvolvida uma ferramenta para
execução de \textit{jobs} paralelos para ambientes multiprocessados, bem como
uma interface gráfica para uso desse ferramental.

A identificação de softwares alternativos ao PAML através do estudo e análise
supracitados resultaram na economia de mais da metade do tempo de análise para
os dados de entrada do grupo de pesquisa do HCPA, enquanto que a ferramenta
desenvolvida para paralelização de \textit{jobs} do SAMtools reduziu de dias
para horas o tempo da análise realizada pelo grupo com essa ferramenta,
permitindo que ela fosse utilizada como alternativa ao software ANGSD, cujos
problemas de desempenho foram mapeados através de um perfil de execução, mas
não foram atacados nesse trabalho, sendo a ferramenta de paralelização do
SAMtools a alternativa favorecida.
% TODO especificar o antes e depois exato do SAMtools

\section{Conceitos de bioinformática}

A fim de compreender o uso e funcionamento dos softwares que foram objeto de
estudo desse trabalho, convém elucidar alguns conceitos de bioinformática. Na
seção \ref{sec:filo} é apresentado o conceito de análise filogenética, objeto
de estudo dos usuários do pacote PAML, na seção \ref{sec:call} é exposto o
conceito de análise de variações genéticas de diferentes populações de uma
espécie através de um processo denominado chamada de variantes, para o qual os
pacotes ANGSD e SAMtools são empregados, e na seção \ref{sec:formats} são
brevemente expostos os formatos de arquivos utilizados para cada um desses
processos.

\subsection{Análise filogenética}
\label{sec:filo}

A análise filogenética, propósito do PAML, compreende o estudo da evolução de
um ou mais organismos e suas características. Nesta sessão são brevemente
apresentados alguns conceitos de análise filogenética relevantes ao
entendimento dos softwares utilizados e seu comportamento.

A informação genética presente no DNA dos seres vivos é utilizada no processo
de síntese proteica, onde os códons (tripla de nucleotídeos) presentes no RNA
mensageiro, gerado a partir do DNA, determinam a síntese de aminoácidos
específicos, componentes das proteínas, macromoléculas fundamentais à vida.

Existem quatro tipos de nucleotídeos no código genético, formados por adenina
(A), guanina (G), uracila (U), e citosina (C). Dessa forma, existem $4^3 = 64$
códons distintos, dos quais três são códons de terminação, que indicam o fim da
etapa de tradução na síntese protéica, enquanto os outros 61 códons traduzem
para um aminoácido específico.

Apenas 20 aminoácidos compõem as proteínas em seres vivos. Sendo assim, a
maioria dos aminoácidos é traduzido por mais de um códon. Em outras palavras, a
substituição de certos códons no DNA não produz alteração nos aminoácidos
gerados na síntese proteica.

As substituições que não geram alterações na síntese proteica são chamadas de
sinônimas ou silenciosas, enquanto as que modificam os aminoácidos gerados são
não-sinônimas. Acredita-se que as substituições sinônimas sejam mais comuns e
não sofram tanta pressão seletiva. A taxa de substituições sinônimas e não
sinônimas $\omega = \frac{dN}{dS}$ é uma medida de seleção natual, conforme a
tabela abaixo.\cite{yang2002codon}

\begin{table}[h]
    \caption{Taxas de substituição}
    % OBS: não use \begin{center}, pois este aumenta o espaçamento entre a caption/legend e a tabela
    % Para figuras, a aparência é melhor com o espaçamento extra
    \centering
        \begin{tabular}{c|c}
          \hline
          \textit{Taxa}  &   \textit{Seleção} \\
          \hline
          \hline
          $\omega = 1$ & Seleção neutra \\
          $\omega < 1$ & Seleção negativa ou purificadora \\
          $\omega > 1$ & Seleção positiva \\
          \hline
        \end{tabular}
      \legend{Fonte: \cite{yang2002codon}}
    \label{tbl:ex1}
\end{table}

Em síntese, o propósito de softwares de análise filogenética como o pacote PAML
é detectar eventos de seleção positiva nas diferentes linhagens da árvore
filogenética de uma espécie. Para isso, softwares como o PAML utilizam modelos
estatísticos de máxima verossimilhança para comparar a hipótese de ocorrência
desses eventos contra a hipótese nula de seleção neutra ou
negativa.\cite{moretti2012gcodeml}

\subsection{Chamada de variantes}
\label{sec:call}

Um dos objetos de estudo da bioinformática é a compreensão dos genes envolvidos
em variações fenotípicas observadas entre diferentes populações de uma
espécie, em particular da espécie humana.\cite{jiang2019population} Para isso,
uma possibilidade é o estudo a nível molecular das variações genéticas entre
indivíduos de diferentes populações. A identificação dessas variações é
denominada ``chamada de variantes'' ou \textit{SNV calling}, o pacote SAMtools
providenciando ferramental para realização desse
proceso.\cite{pirooznia2014validation} Existem diversos métodos para
identificação e análise dessas variações, um desses métodos sendo o
\textit{Population Branch Statistic} ou PBS,\cite{jiang2019population} para o
qual pode ser empregado o software ANGSD.

Uma métrica de diferença na estrutura genética entre duas populações de uma
espécie é o índice de fixação ou $F_{ST}$, baseado, em suma, na variança da
frequência alélica das duas populações. Genes com um $F_{ST}$ alto são
potenciais alvos de seleção natural.\cite{yi2010sequencing} Como o $F_{ST}$ é
uma métrica de comparação de pares, ele não é capaz de identificar a direção
das mudanças, nem pode ser utilizado como única métrica para identificar
eventos de seleção natural. O PBS consiste de um método estatístico que
utiliza comparações em pares do $F_{ST}$ entre três populações distintas para
quantificar as diferenças entre suas sequências genéticas. Genes com valor
PBS alto indicam seleção positiva.\cite{jiang2019population} Uma descrição
mais aprofundada do método foge ao escopo desse trabalho, e pode ser
encontrada em \cite{yi2010sequencing}.

Os pesquisadores do HCPA utilizam os pacotes ANGSD e SAMtools para realizar
a chamada de variantes visando entender variações fenotípicas entre populações
humanas, utilizando para isso o método PBS.

\subsection{Formatos de arquivos}
\label{sec:formats}

Fundamental ao entendimento de algumas otimizações realizadas nesse trabalho é
um conhecimento básico dos diferentes formatos de arquivo utilizados para
armazenamento de informação genética, no campo da bioinformática. Neste
trabalho foram trablhados com arquivos nos formatos SAM, BAM, CRAM, BCF, VCF, e
FASTA, descritos brevemente nesta seção.

Os formatos SAM, BAM, e CRAM, representam sequências genéticas alinhadas a um
genoma de referência. O formato SAM traz uma representação em texto-plano, o
BAM é seu equivalente binário, e o CRAM é uma versão binária com compressão.
Tais arquivos podem ser manipulados com a ferramenta samtools.\cite{danecek2021twelve}

A chamada de variantes realiza comparações entre múltiplos arquivos nos
formatos supracitados, produzindo arquivos que representam as variantes
encontradas. Quando essas informações são armazenadas em texto-plano diz-se do
arquivo VCF, seu equivalente binário sendo o formato BCF, ambos gerados e
manipulados pela ferramenta bcftools.\cite{danecek2021twelve}

Convém mencionar ainda o formato FASTA, utilizado para representar sequências
de nucleotídeos ou aminoácidos. Introduzido pela primeira vez em
\cite{doi:10.1126/science.298342}, é hoje ubíquoto no campo da
bioinformática.\cite{shen2016seqkit} Os genomas de referência aos quais as
demais sequências genéticas são alinhadas são fornecidos nesse formato, e o
pacote SAMtools providencia ferramental para manipulação desses arquivos.

% TODO falar dos formatos do PAML

\chapter{Pacote PAML}

O \textit{Phylogenetic Analysis by Maximum Likelihood} (PAML) é um pacote de
software com ferrametas para análise filogenética utilizando métodos
estatísticos de máxima verossimilhança.\cite{yang2007paml} Dentre
outras ferramentas disponíveis no pacote se destaca o codeml,
amplamente utilizado na literatura.\cite{maldonado2016lmap} Apesar de
estatisticamente robusto,\cite{maldonado2016lmap} o codeml possui
implementação ingênua de métodos numéricos computacionalmente
custosos.\cite{yang2020paml} Para o caso de uso do grupo de pesquisa em
genética do HCPA, o tempo de execução é de vários dias.

Neste trabalho foi realizada uma revisão bibliográfica a respeito do software
em questão e de alternativas a ele; foram realizadas análises de seu desempenho
através de perfis de execução, identificando as rotinas responsáveis pela maior
parte do tempo de execução; os garagalos de desempenho foram estudados através
de análise do código fonte e do manual da ferramenta, em seguida foi realizada
revisão bibliográfica a respeito dos métodos numéricos empregados, visando sua
otimização. Foram implementadas estratégias de paralelização de tais métodos e
realizadas novas análises de desempenho e de corretude do software
paralelizado. Por fim, foram realizadas análises de desempenho de softwares
alternativas encontrados no estudo da literatura.

\section{Trabalhos anteriores}

Em \cite{moretti2012gcodeml} os autores exploram o fato de múltiplas execuções
do codeml serem independentes, o que torna a aplicação embaraçosamente
paralela para múltiplas entradas. Nesse trabalho os autores visam atender às
necessidades do grupo de pesquisa que mantém o banco de dados Selectome, em
que múltiplas instâncias do codeml precisam ser executadas, uma para cada
arquivo de entrada mantido pelo banco de dados. Para isso, os autores obtam
por um paralelismo de \textit{jobs}, desenvolvendo uma ferramenta voltada
para execução em cluster, mais especificamente visando ambientes com o
\textit{middleware} Advanced Resource Connector (ARC), utilizado no grid aos
quais os autores possuíam acesso. A ferramenta, chamada gcodeml, é desenvolvida
na linguagem de programação Python, utilizando a biblioteca GC3Pie, que
providencia \textit{bindings} para controle e execução de \textit{jobs} ARC. O
caso de uso dos pesquisadores do HCPA envolve um arquivo de entrada único, e
o ambiente de execução aos quais os pesquisadores possuem acesso não é de um
cluster, portanto essa solução não foi estudada mais a fundo nesse trabalho.

Em \cite{maldonado2016lmap} é implementado uma
ferramenta para execução paralela de múltiplos \textit{jobs} do codeml
em uma única máquina (em CPU). % TODO expandir

Em \cite{schabauer2012slimcodeml} os autores otimizam e organizam software
original (codeml), mantendo o formato de entrada e saída bem como os métodos
numéricos utilizados. Em mais detalhes, os autores substituem implementações
ingênuas de métodos numéricos por aquelas de bibliotecas amplamente utilizadas
na literatura como BLAS e LAPACK, além de manipularem equações matriciais a fim
de permitir o uso de implementações mais eficientes mas dependentes de
propriedades como a simetria das matrizes envolvidas na operação. A
implementação é sequencial. O novo software, nomeado slimcodeml, desempenhou
quase dez vezes melhor que o original para os dados de entrada dos autores, em
uma máquina com processador Intel Xeon W3540 de 2.93 GHz. O fato da entrada e
saída serem os mesmos, bem como os métodos numéricos, e o bom desempenho em
ambiente semelhante ao utilizado pelos pesquisadores do HCPA fizeram dessa
ferramenta um dos focos de avaliação de desempenho desse trabalho.

Em \cite{valle2014optimization} os mesmos autores do slimcodeml escrevem um
novo software, com base de código completamente independente do codeml
original, fornecendo uma implementação paralela em CPU. Essa implementação,
chamada fastcodeml, possui formatos de entrada e saíde diferentes do software
original, e implementa apenas um sub-conjunto de seus métodos. O propósito dos
autores nesse trabalho foi explorar estratégias de paralelização e otimização
para o problema de análise filogenética de forma geral. Não possuindo as mesmas
entradas e saídas do software original nem implementando todos seus métodos,
essa solução não é útil aos pesquisadores do HCPA, mas serve como referência
para o desenvolvimento de novos softwares de análise filogenética.

Nesse trabalho foram exploradas as soluções supracitadas, com ênfase na análise
do desempenho da ferramenta desenvolvida por \cite{schabauer2012slimcodeml},
por ser a que melhor atendia aos requisitos dos pesquisadores do HCPA.

\section{Análise de desempenho}

A primeira aplicação a ser testada foi o
slimcodeml.\cite{schabauer2012slimcodeml} Utilizando os dados de entrada
fornecidos por pesquisadores do grupo, obteve-se uma redução no tempo total de
execução de 16h38m para 5h24m, ou 67,53\%, em relação ao codeml original. Os
testes foram realizados em um ambiente controlado, de uso exclusivo dos
autores, em uma máquina com o hardware descrito na tabela \ref{tbl:thor1},
doravante ``Thor1''.

\begin{table}[h]
    \caption{Hardware da máquina ``Thor1''}
    \centering
        \begin{tabular}{c|c}
          \hline
          \textit{Componente}  &   \textit{Especificação} \\
          \hline
          \hline
          Modelo CPU & Intel Core i5-9400 \\
          Cores CPU & 6 núcleos\\
          Clock CPU & 2.90 GHz (base), 4.5 GHz (turbo) \\
          Cache CPU & 9M \\
          RAM & 64 GB \\
          \hline
        \end{tabular}
      \legend{Fonte: Os autores}
    \label{tbl:thor1}
\end{table}

Em um segundo momento foi testado a aplicação LMAP.\cite{maldonado2016lmap}
% TODO resultados dos testes da LMAP

A tabela \ref{tbl:paml} apresenta uma comparação dos tempos de execução de cada
uma das ferramentas estudadas para os arquivos de entrada fornecidos pelos
pesquisadores do HCPA, no ambiente de execução descrito na tabela \ref{tbl:thor1}.

\begin{table}[h]
    \caption{Tempos de execução das ferramentas de análise filogenética}
    \centering
        \begin{tabular}{c|c}
          \hline
          \textit{Ferramenta}  &   \textit{Tempo de execução} \\
          \hline
          \hline
          codeml (PAML) & 16h38m \\
          slimcodeml & 5h24m \\
          codeml (LMAP) & TODO \\
          \hline
        \end{tabular}
      \legend{Fonte: Os autores}
    \label{tbl:paml}
\end{table}

\section{Perfil de execução e estudo de implementação paralela}
\label{subsec:codemlpar}

Além do estudo de implementações presentes na literatura, nesse trabalho o
codeml foi perfilado utilizando as ferramentas callgrind e
kcachegrind,\cite{weidendorfer2008sequential} para os dados de entrada do caso
de uso do grupo de pesquisa do HCPA, objetivando-se identificar gargalos de
desempenho e explorar soluções alternativas às presentes na literatura.

Com base no resultado desse perfil de execução, foram implementadas e testadas
algumas abordagens de paralelismo em CPU das rotinas que consumiam o maior
tempo de execução do codeml, mas tal abordagem foi posteriormente descartada em
favor das soluções encontradas na literatura, conforme será descrito abaixo.
Uma abordagem para GPU foi estudada, mas igualmente descartada em favor das
soluções existentes.

A fim de determinar os gargalos de desempenho do codeml para o caso de uso do
grupo de pesquisa do HCPA, foi traçado um perfil de execução da aplicação com
os dados de entrada fornecidos pelos pesquisadores do HCPA utilizando as
ferramentas callgrind e kcachegrind.\cite{weidendorfer2008sequential}

O perfil de execução revelou que $97,28\%$ do tempo de execução da aplicação
era dispendido na rotina \textit{ming2}. Um estudo do código fonte revela que
ela implementa o algoritmo de Broyden–Fletcher–Goldfarb–Shanno (BFGS), um
método numérico para resolução de problemas de otimização. Tal método possui,
no caso do codeml, duas sub-rotinas responsáveis por quase a totalidade de seu
tempo de execução: \textit{gradientB} e \textit{LineSearch2}.

A rotina \textit{gradientB}, responsável por $42,99\%$ do tempo total da
aplicação, implementa o cálculo do gradiente via diferenças finitas, enquanto
\textit{LineSearch2}, responsável por $53,27\%$ do tempo de execução,
implementa um método numérico de busca linear utilizando interpolação
quadrática, descrito em \cite{wolfe1978numerical}. A
figura~\ref{fig:kcachegrind} fornece uma visualização da pilha de chamadas em
questão.

\begin{figure} \caption{Pilha de chamadas do codeml} \begin{center}
\includegraphics[width=0.3\linewidth]{img/kcachegrind.png} \end{center}
\legend{Fonte: Os Autores} \label{fig:kcachegrind} \end{figure}

Foi considerada uma implementação paralela para todos os métodos numéricos
supracitados. Em um primeiro momento o cálculo do gradiente foi paralelizado
utilizando OpenMP, um modelo de programação paralela para sistemas com
múltiplos processadores com memória compartilhada.\cite{chandra2001parallel} A
escolha de OpenMP se dá por três fatores: ser adequado ao ambiente de trabalho
dos pesquisadores do HCPA (máquinas multiprocessadas), pela simplicidade de uso
para paralelização de laços, e pela disponibilidade nos ambientes utilizados.

O cálculo do gradiente é aproximado utilizando diferenças finitas para obter as
derivadas parciais de primeiro grau. O codeml permite utilizar diferenças
finitias progressivas, centradas, ou regressivas. A paralelização se dá sob
todas variáveis da função cujo gradiente está sendo obtido. O número de threads
foi definido pelo mínimo entre o número de núcleos de processamento disponíveis
e o número de variáveis na função (iterações no laço). Foi utilizado um
escalonador estático com tamanho do bloco igual ao número de variáveis sob o
número de threads, uma vez que a carga de trabalho é homogênea (as variáveis
são todas da mesma função). O algoritmo encontra-se reproduzido abaixo, onde
$p$ é o número de processadores disponíveis, $t$ o número de threads a ser
usado, $c$ o tamanho do bloco, e $n$ o número de variáveis.

\begin{algorithmic}
\State $t \gets \max(1, \min(p, n))$
\State $c \gets \max(1, \frac{n}{t})$
\For{$i \gets 0,n$ \textbf{in parallel}}
  \If{centrada}
    \State $\frac{\partial f}{\partial x_i} \gets \frac{f(x+h)-f(x-h)}{2h}$
  \ElsIf{progressiva}
    \State $\frac{\partial f}{\partial x_i} \gets \frac{f(x+h)-f(x)}{h}$
  \Else
    \State $\frac{\partial f}{\partial x_i} \gets \frac{f(x)-f(x-h)}{h}$
  \EndIf
\EndFor
\end{algorithmic}

A fim de garantir a corretude da implementação paralela foram desenvolvidos
testes unitários para a função. Em um primeiro momento os testes foram
executados tomando como entrada uma função arbitrária com derivada conhecida,
e os resultados da implementação paralela comparados com aqueles obtidos
simplesmente chamando a função derivada \textit{a priori}. Uma vez que esse
teste foi bem sucedido, testou-se como entrada a função sendo derivada na
implementação do codeml.

Os testes revelaram que a implementação paralela gerava resultados diferentes
da sequencial para a função sendo diferenciada no codeml, mas não para funções
arbitrárias com derivada conhecida. O problema consistia da função sendo
diferenciada ter sido implementada de forma não \textit{thread-safe} pelos
autores do codeml, gerando condições de corrida que inviabilizam a simples
paralelização da rotina em questão.

Foi considerado a re-implementação das funções sendo diferenciadas, todavia, o
uso extensivo de variáveis globais e memória compartilhada na implementação
original do codeml demonstrou-se um grande obstáculo para essa abordagem, que
por isso foi abandonada. Foi estudada então a implementação do slimcodeml, que
re-organiza amplas seções do código fonte, na esperança de que tais
dependências pudessem ter sido removidas, possibilitando a implementação
paralela do cálculo do gradiente. Todavia, apesar de melhor organizado, o
slimcodeml ainda apresenta o mesmo compartilhamento de memória e uso de
variáveis globais encontrados na implementação original, inviabilizando essa
estratégia de paralelização nesse software.

A partir daí foram voltadas as atenções para \textit{LineSearch2}, mas não foi
encontrado na literatura implementações paralelas para o método descrito em
\cite{wolfe1978numerical}, o algoritmo não é de paralelização trivial, e
funções auxiliares utilizadas em seu cálculo apresentavam os mesmos problemas
de compartilhamento de memória encontrados nas outras rotinas estudadas, tanto
na implementação original como no slimcodeml, portanto a paralelização dessa
rotina também foi abandonada.

Por fim, as atenções foram voltadas para o próprio \textit{ming2}, cujo método
numérico (BFGS) é iterativo. Foi realizada uma revisão bibliográfica, que
revelou implementações paralelas desse algoritmo para GPU em
\cite{fei2014parallel}. Os resultados mostram que a implementação não
performa bem para entradas pequenas. Foi realizado então um estudo do uso pelo
codeml desse algoritmo, realizando para isso pequenas adaptações no código para
aumentar a verbosidade dos \textit{logs} da aplicação, e foi observado que o
codeml trabalha com uma entrada fixa de tamanho 61 para esse algoritmo -- o
número códons que traduzem para aminoácidos -- tamanho esse considerado pequeno
com base no estudo supracitado. Dessa forma, conclui-se que uma paralelização
em GPU seria ineficiente. Além disso, a paralelização desse algoritmo é
complexa e de difícil adaptação ao caso do codeml. Dessa forma, essa abordagem
também foi descartada.

Esgotadas todas possibilidades de paralelização das rotinas responsáveis por
quase a totalidade do tempo de execução do codeml para o caso de uso dos
pesquisadores do HCPA, optou-se por fornecer aos pesquisadores as ferramentas
encontradas através de estudo da literatura. Em particular, os pesquisadores
mostraram-se satisfeitos com o uso do slimcodeml como uma alternativa à
implementação original do codeml.

\chapter{Pacote ANGSD}
\label{sec:angsd}

Outra aplicação utilizada pelo grupo de pesquisa em genética é o pacote ANGSD,
software para análise genética de diferentes populações de uma
espécie.\cite{korneliussen2014angsd} Os pesquisadores utilizam esse software
para determinar se a genética de uma população influencia em alguma
característica específica de seus indivíduos, através de um teste chamado
\textit{Population Branch Statistic} (PBS), caracterizado pela comparação da
frequência de ocorrência de determinados alelos entre pares de
indivíduos de diferentes populações.\cite{yi2010sequencing}

A análise em questão consiste de duas etapas, a primeira utilizando o binário
angsd, a aplicação principal do pacote ANGSD, e a segunda etapa utilizando a
ferramenta realSFS, um utilitário fornecido pelo pacote. A primeira etapa leva
cerca de um dia para cada arquivo de entrada e porventura apresenta uso elevado
de memória, enquanto que a segunda etapa, que recebe como entrada a saída da
etapa anterior, nunca termina a execução devido a uso de memória
proibitivamente elevado, para os dados de entrada dos pesquisadores do HCPA.

A entrada da primeira etapa consiste em arquivos nos formatos BAM ou CRAM,
descritos na seção \ref{sec:formats}, um arquivo de entrada para cada indivíduo
de cada população sob análise. O angsd então utiliza a biblioteca
htslib\cite{bonfield2021htslib} para manipulação desses arquivos.

Uma primeira observação realizada através de reuniões com os pesquisadores é
que esses dados de entrada são obtidos sempre no formato CRAM do projeto 1000
Genomes, um esforço colaborativo internacional que visa sequenciar o genoma
humano e disponibilizar a informação publicamente,\cite{via20101000} e que
estava sendo realizada uma etapa de conversão de CRAM para BAM para uso do
angsd. Essa conversão é realizada utilizando o software SAMtools, descrito em
mais detalhes na seção \ref{sec:SAMtools}, e leva algumas horas para cada
arquivo de entrada. Todavia, como observado anterioremente, o angsd trabalha
também com o formato CRAM, fato esse observado através do estudo do manual da
ferramenta.  Sendo assim, a etapa de conversão é desnecessária. Essa observação
já economizou algumas horas de execução.

Eliminada essa etapa de pré-processamento dos dados, realizou-se um estudo do
software a fim de elucidar seu funcionamento, para posteriormente realizar um
perfil de seu uso de memória com a ferramenta massif, do software
callgrind.\cite{weidendorfer2008sequential}

Inicialmente realizou-se um estudo do utilitário realSFS, que apresentava o
principal problema de desempenho. Através de um estudo do código fonte foi
observado que o utilitário implementa dois otimizadores, um legado utilizando
BFGS (o mesmo algoritmo usado no pacote PAML) e o padrão utilizando
``Electromagnetism-like Mechanism'', um método estocástico de otimização
não-linear que utiliza mecanismos de atração e repulsão para mover
''partículas`` na direção ótima.\cite{5636954} Foi observado ainda que,
independente do otimizador utilizado, a alocação de memória é a mesma. Fatores
que influenciam no uso de memória incluem o número de threads e o número de
síteos considerados. Em ambos os casos há um \textit{trade-off} - ao reduzir o
número de threads o tempo de execução aumenta, e ao reduzir o número de síteos
a confiabilidade dos resultados diminui.\cite{popgen2016angsd} A redução de
ambos é indesejável, visto que o tempo de execução já é muito elevado e a
confiabilidade dos resultados é fundamental.

Foi traçado então um perfil de memória da aplicação utilizando callgrind,
visando elucidar as causas do uso elevado de memória. Esse perfil revelou que a
etapa de otimização supracitada é o principal responsável pelo uso
proibitivamente alto de recursos, e não foram encontradas oportunidades de
melhoria que não afetassem o tempo de execução ou a confiabilidade dos
resultados.

Foi realizada uma reunião com os pesquisadores, em que foi observado que é
possível realizar a mesma análise feita com o angsd (PBS) utilizando as
ferramentas SAMtools e bcftools, construídas em cima da biblioteca htslib (tal
qual o angsd), e que já eram utilizadas na etapa de pré-processamento dos
dados, seguida de uma análise estatística dos resultados. Todavia, o pipeline
de análise utilizando tais ferramentas apresentava tempos proibitivamente
lentos. Na seção \ref{sec:SAMtools} é descrito um estudo desse pipeline e o
posterior desenvolvimento de uma ferramenta que otimiza e paraleliza sua
execução, resolvendo o problema de desempenho supracitado. Tal estudo foi
realizado em paralelo ao estudo do angsd descrito acima e, uma vez que os
resultados foram se mostrando mais frutíferos, o foco dese trabalho passou a
ser esse ferramental, a análise pelo angsd sendo substituída pela análise pelo
SAMtools por parte dos pesquisadores.

\chapter{Pacote SAMtools}
\label{sec:SAMtools}

Conforme mencionado na seção \ref{sec:angsd}, uma aplicação utilizada pelo
grupo de pesquisa em genética é o pacote SAMtools,\cite{li2009sequence}
aplicação amplamente utilizada na literatura para análise de sequências
genéticas.\cite{danecek2021twelve} O samtoools e a ferramenta bcftools que o
acompanha são construídos em cima da biblioteca htslib, dos mesmos autores, que
permite a leitura e manipulação de arquivos de arquivos em formatos diversos
representando sequências genéticas alinhadas, descritos em \ref{sec:formats}.
Dentro outras funcionalidades dessas ferramentas destaca-se a chamada de
variantes, processo de comparação de sequências genéticas
alinhadas.\cite{danecek2021twelve}

Em mais detalhes, a aplicação samtools providencia o comando \textit{view} para
conversão entre formatos de arquivo, filtragem, e extração de porções de um
arquivo, enquanto comandos como \textit{sort} e \textit{merge} permitem
reordenar e agrupar os arquivos de diversas formas, e os comandos
\textit{index} e \textit{faidx} indexam os arquivos para acesso aleatório
rápido. A ferramenta providencia uma série de outras funcionalidades que fogem
ao escopo desse trabalho e estão descritas em \cite{danecek2021twelve}.

Já a aplicação bcftools, parte do pacote SAMtools, providencia comandos para
execução da chamada de variantes -- \textit{mpileup} e \textit{call}, que
calculam as variantes entre as sequências alinhadas lidas e agrupadas através
de um processo descrito em \cite{li2011improving}. O bcftools providencia mais
21 comandos com mais de 230 opções diferentes para diversas análises das
sequências genéticas alinhadas manipuladas pelo samtools. Uma descrição
completa desses comandos foge ao escopo desse trabalho e pode ser encontrada em
\cite{danecek2021twelve}.

Nesse trabalho foi estutdado, otimizado, e paralelizado o pipeline de análise
do grupo de pesquisa do HCPA utilizando o pacote SAMtools para chamada de
variantes, reduzindo o tempo de execução de vários dias para poucas horas, e
desenvolvida uma interface gráfica para execução do pipeline otimizado.

\section{Fluxo de análise original}

A fim de realizar a chamada de variantes, os pesquisadores primeiro obtém as
sequências genéticas alinhadas de múltiplos indivíduos de pelo menos três
populações no formato CRAM, bem como o genoma de referência no formato FASTA,
descritos na seção \ref{sec:formats}. Todos dados são obtidos do projeto 1000
Genomes, um esforço colaborativo internacional que visa sequenciar o genoma
humano e disponibilizar a informação publicamente.\cite{via20101000}

O fluxo de análise original fornecido pelos pesquisadores pode ser dividido em
três etapas, após a obtenção dos dados. Na primeira etapa é executada uma
conversão de formato de CRAM para BAM utilizando o comando \textit{samtools
view -b}, seguida de uma rotina de indexação dos arquivos utilizando
\textit{samtools index}.

Uma vez convertidos e indexados, cada arquivo de entrada é separado em 22
arquivos de saída, um para cada par de cromossomos autossômicos humanos,
utilizando o comando \textit{samtools view chr}. Essa etapa na sua forma
originalmente usada pelo grupo de pesquisa encontra-se reproduzida na
figura~\ref{fig:stage1_orig}.

\begin{figure}
  \caption{Fluxo de análise original via SAMtools, etapa 1}
    \begin{center}
      \includegraphics[width=0.85\linewidth]{img/stage1_orig.png}
    \end{center}
    \legend{Fonte: Os Autores}
    \label{fig:stage1_orig}
\end{figure}

Numa segunda etapa, para cada cromossomo são aglutinados os respectivos
arquivos de todos indivíduos, utilizando o comando \textit{SAMtools merge}.
Posteriormente esses arquivos são indexados, e os comandos \textit{bcftools
mpileup} e \textit{bcftools call} são invocados para executar a chamada de
variantes.

Por fim, o comando \textit{bcftools view} é executado para converter a saída do
formato BCF para VCF. Independente do número de arquivos de entrada, a saída é
sempre 22 arquivos. Essa etapa encontra-se reproduzida na
figura~\ref{fig:stage2_orig}.

\begin{figure}
  \caption{Fluxo de análise original via SAMtools, etapa 2}
    \begin{center}
      \includegraphics[width=0.85\linewidth]{img/stage2_orig.png}
    \end{center}
    \legend{Fonte: Os Autores}
    \label{fig:stage2_orig}
\end{figure}

A última etapa consiste de concatenar todos arquivos gerados na etapa anterior
e realizar uma série de filtros utilizando a ferramenta VCFtools.\cite{10.1093/bioinformatics/btr330}

\section{Fluxo de análise otimizado}

Através de um estudo da literatura e do manual das ferramentas utilizadas
percebeu-se oportunidade de melhorias no fluxo de análise utilizado pelos
pesquisadores. Em particular, em,\cite{danecek2021twelve} os autores observam
que o uso do comando \textit{samtools view -b}, para conversão de CRAM para
BAM, apesar de frequentemente presente em guias na internet é normalmente
desnecessário, uma vez que as ferramentas possuem suporte a CRAM. Isso ocorre
por motivos históricos, pois originalmente o SAMtools não trabalhava com
arquivos CRAM.\cite{danecek2021twelve}

Foram realizados testes de desempenho de cada uma das etapas do fluxo original
de análise, e a conversão de CRAM para BAM era particularmente custosa.
Observado isso, foi testado e comprovado que todo restante da análise poderia
ser executado sem essa etapa, observando o suporte para CRAM e comparando as
saídas do fluxo otimizado com o fluxo original.

Além disso, foi observado que ambas etapas eram embaraçosamente paralelas,
consistindo de passos independentes para cada arquivo de entrada. Como os
pesquisadores do HCPA possuem acesso a máquinas multiprocessadas, foi realizada
uma paralelização de ambas etapas utilizado GNU Parallel\cite{tange2011gnu}. A
escolha dessa ferramenta se deu pela facilidade de uso, disponibilidade nos
ambientes usados pelos pesquisadores, e suporte a ajuste automático do número
de \textit{jobs} paralelos ao número de núcleos de processamento disponíveis.

Foi desenvolvida uma ferramenta para execução do fluxo de análise otimizado e
paralelizado. Um pseudo-código da ferramenta encontra-se reproduzido abaixo.

\begin{algorithmic}
  \State $C \gets \text{input CRAMs}$
  \State $n \gets \text{input length}$
\For{$j \gets 0,n$ \textbf{in parallel}}
  \State SAMtools index C(j)
  \For{$i \gets 1,22$}
    \State $R(i,j) \gets \text{SAMtools view } C(j) \text{ chr i}$
  \EndFor
\EndFor

\For{$i \gets 1,22$ \textbf{in parallel}}
  \State $M(i) \gets \text{SAMtools merge } R(i,j)$ \textbf{for j in 0,n}
  \State SAMtools index $M(i)$
  \State bcftools mpileup $M(i)$
  \State bcftools call $M(i)$
  \State bcftools view $M(i)$
\EndFor
\end{algorithmic}

As otimizações e a paralelização reduziram drasticamente o tempo total de
execução da análise. Na tabela \ref{tbl:SAMtools} encontram-se reproduzidos os
tempos de execução utilizando o pipeline original e o otimizado. Os testes
foram realizados na máquina Thor1, descrita na tabela \ref{tbl:thor1},
utilizando todos núcleos de processamento disponíveis.

\begin{table}[h]
    \caption{Tempos de execução do SAMtools}
    \centering
        \begin{tabular}{c|c|c}
          \hline
          \textit{Tamanho da entrada}  &   \textit{Tempo original}  & \textit{Tempo otimizado} \\
          \hline
          \hline
          58 GB (2 CRAMs) & TODO & TODO \\
          124 GB (4 CRAMs) & TODO & TODO \\
          232 GB (8 CRAMs) & TODO & TODO \\
          \hline
        \end{tabular}
      \legend{Fonte: Os autores}
    \label{tbl:SAMtools}
\end{table}

Além disso, foram realizados testes de desempenho para a entrada de 58GB
utilizando um número diferente de núcleos de processamento, a fim de estimar o
\textit{speedup} bem como a quantidade de tempo economizado pela paralelização,
o resto sendo atribuído às otimizações no fluxo de análise. Os resultados
encontram-se reproduzidos na tabela \ref{tbl:speedup}.

\begin{table}[h]
    \caption{Speedup da ferramenta}
    \centering
        \begin{tabular}{c|c|c}
          \hline
          \textit{Número de núcleos}  &   \textit{Tempo de execução}  & \textit{Speedup} \\
          \hline
          \hline
          1 núcleo & TODO & TODO \\
          2 núcleos & TODO & TODO \\
          4 núcleos & TODO & TODO \\
          6 núcleos & TODO & TODO \\
          \hline
        \end{tabular}
      \legend{Fonte: Os autores}
    \label{tbl:speedup}
\end{table}

\section{Ambiente e interface}

Tanto o SAMtools e o bcftools como a ferramenta desenvolvida pelos autores
requer um ambiente com uma série de softwares pré-instalados e em versões
específicas, o que porventura gerava dificuldades aos pesquisadores do HCPA.

A fim de superar tais dificuldades, bem como disponibilizar um ambiente para
reprodução fidedigna dos resultados obtidos nesse trabalho, decidiu-se por
utilizar o software Docker, de virtualização a nível de sistema operacional,
que fornece uma série de vantagens para pesquisa
reprodutível.\cite{boettiger2015introduction} A imagem criada está disponível
em repositório público.\cite{dockerme}

Além disso, foi desenvolvida uma interface gráfica para o uso do SAMtools e
bcftools para chamada de variantes. O intuito é facilitar o uso do ferramental
para profissionais da biologia que porventura não estejam familiariezados com
interfaces por linha de comando.

A interface é executada na máquina local do usuário, que pode configurar uma
máquina remota para execução do pipeline de análise. Além da execução do
pipeline, a interface automatiza uma série de outros passos realizados pelos
pesquisadores, como obtenção dos dados e manipulação dos arquivos de resultado.
A interface permite ainda realizar todas etapas em segundo plano, verificar seu
progresso, e interrompê-las.

Para desenvolvimento da interface obtou-se por utilizar a linguagem de
programação Python, amplamente utilizada para computação
científica.\cite{oliphant2007python} A escolha da linguagem se deu
principalmente pelo sua disponibilidade em múltiplos sistemas operacionais,
visto que os usuários (pesquisadores do HCPA) manifestaram interesse em
utilizar a ferramenta nos sistemas operacionais Windows, Linux, e Mac, e há
suporte tanto de interpretadores da linguagem como das bibliotecas utilizadas
para todas essas plataforams.\cite{oliphant2007python} Outro fator que
influenciou a escolha foi a agilidade no desenvolvimento que a linguagem
proporciona.\cite{oliphant2007python}

Foi utilizada a biblioteca \textit{guietta} para criação da interface
gráfica,\cite{guietta} um \textit{wrapper} em cima da biblioteca
\textit{PySide2}, um \textit{binding} para Python da biblioteca multiplataforma
Qt.\cite{loganathan2013pyside} A escolha da bilbioteca vem pela sua facilidade
de uso e pela disponibilidade multiplataforma. A figura \ref{fig:samgui}
apresenta uma captura de tela da interface gráfica desenvolvida.

\begin{figure}
  \caption{Captura de tela da interface gráfica ``SAMGUI''}
    \begin{center}
      \includegraphics[width=0.85\linewidth]{img/samgui.png}
    \end{center}
    \legend{Fonte: Os Autores}
    \label{fig:samgui}
\end{figure}

\bibliographystyle{abntex2-alf}
\bibliography{biblio}

\end{document}
